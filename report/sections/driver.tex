\section{Driver}
The driver is the agent responsible to bridge the high level user interface offered by CryptIC to the low level kernel operations.

It is internally organized in two interface-like modules, each implementing a portion of the core functionalities, and the top-level driver to initialize the submodules and to implement the standard Linux driver facilities. The two submodules are the Cryptographic Interface Driver to integrate CryptIC with the Linux Cryptographic API, and the USB Driver to handle USB requests and communications.

\subsection{Cryptographic Interface Driver}
\subsection{USB Driver}
This module, named ``crypticusb'' for simplicity, is inspired from the official documentation and the examples available from the \href{http://www.linux-usb.org/}{official Linux USB documentation} and from \href{https://github.com/torvalds/linux/blob/master/drivers/usb/usb-skeleton.c}{Linus Torvalds' implementation} of a skeleton USB driver on GitHub, hereon referred to as the ``skel driver''.

\paragraph{Driver Type} Crypticusb is a character device driver, meaning it interacts with a hardware device through character-by-character and relatively small data exchanges. Unlike most device drivers, crypticusb does not offer \textit{file operations}. That is because this driver, despite being a standard USB driver with no knowledge of cryptography, is meant to be under the sole control of the CryptIC driver. Moreover, it is also a bad idea to let any user access a public file containing information on supposedly secure data transfers.

\paragraph{User Interface} Crypticusb instead offers a programming interface to its implementations of USB data transfers. Its header contains the declaration of fucntions which are relevant to the user:
\begin{lstlisting}
/* USB module setup */
int crypticusb_init(void);
void crypticusb_exit(void);

/* USB module interface */
ssize_t crypticusb_send(const char *buffer, size_t count);
ssize_t crypticusb_read(char *buffer, size_t count);
int crypticusb_isConnected(void);
\end{lstlisting}
The functions \texttt{crypticusb\_init()} and \texttt{crypticusb\_exit()} initialize and de-initialize the driver. Specifically, they register and un-register the driver to the kernel and provide it with some useful information, such as what devices should be controlled by this driver and what functions to call upon certain events. For instance, device connection and disconnection are handled by \texttt{crypticusb\_probe()} and \texttt{crypticusb\_disconnect()} which are automatically called by the kernel when their associated events happen. \\

\paragraph{Initialization} The registration process consists in calling standard Linux functions and passing them specific \texttt{struct}s. To give an example, below is a snippet of one of these structs and the implementation of \texttt{crypticusb\_init()}:
\begin{lstlisting}
static struct usb_driver crypticusb_driver = {
        .name = CRYPTIC_DEV_NAME,
        .probe = crypticusb_probe,
        .disconnect = crypticusb_disconnect,
        .id_table = crypticusb_devs_table
};

...

int crypticusb_init(void) {
    int status;
    /* Register driver within USB subsystem */
    status = usb_register(&crypticusb_driver);
    if (status != 0) {
        pr_err(CRYPTIC_DEV_NAME ": could not register USB driver: error %d\n", status);
        return -1;
    }
    pr_info(CRYPTIC_DEV_NAME ": succesfully registered USB driver!\n");
    return 0;
}
\end{lstlisting}

\paragraph{Data Transfers} \texttt{crypticusb\_send()} and \texttt{crypticusb\_read()} respectively \emph{send} and \emph{receive} \texttt{count} bytes to the USB device and they both return either the effective number of transmitted bytes - a positive number - or a negative error code in case of problems. The actual heavy lifting is internally managed by the functions offered by the Linux USB subsysten: what these functions have to do is to check for errors, handle the \textit{mutexes} to prevent race conditions on the serial port and to allocate and initialize a URB, the USB Request Buffer.

\subsection{CryptIC Driver}
\subsection{Makefile}
\subsection{Installer}
